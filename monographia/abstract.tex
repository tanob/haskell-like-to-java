% abstract.tex
\begin{abstract}

\noindent
ALVES, Adriano B. \textbf{Compilando c�digo funcional para \textit{bytecodes} Java}. 2008. 73f.
Trabalho acad�mico (Gradua��o) -- Bacharelado em Ci�ncia da Computa��o. Universidade Federal de Pelotas, Pelotas.
\\

\onehalfspacing

\noindent
This work proposes the study of interoperability and problems involved in the translation between 
languages from different paradigms, particularly the functional paradigm and the object oriented.
For this study, was developed a compiler of a functional language that generates Java bytecodes. 
The interoperability between languages is of great interest, as demonstrated by the success of the platforms .NET 
and Java, respectively of Microsoft and Sun Microsystems. 
One of the reasons for the low adoption of functional languages is the lack of an easy way these interoperate 
with other languages. 
The Java virtual machine runs programs in a binary form called Java bytecodes. 
Although this virtual machine has been developed for the Java programming language, other programming
languages can benefit themselves too.
Some of the reasons for the great interest in virtual machines are the explosion of the web, the development
of large systems based on several languages and the support for legacy code. The virtual machine solves these 
problems because it is a platform where different languages can interoperate. 
The functional language defined in this work is similar to the syntax of Haskell, 
is strict, statically typed, has type inference and does not have support to overload.
\\

\noindent
Keywords: Type system. Functional languages. Interoperability. Object Oriented.
\end{abstract}
