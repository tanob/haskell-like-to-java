% resumo.tex
\keyword{sistema de tipos}
\keyword{linguagens funcionais}
\keyword{interoperabilidade}
\keyword{orienta��o a objeto}

\begin{abstract}

Este trabalho prop�e o estudo da interoperabilidade e dos problemas envolvidos na tradu��o entre 
linguagens de diferentes paradigmas, principalmente entre os paradigmas funcional e orientado a objeto.
Para a realiza��o deste estudo, foi desenvolvido um compilador de uma linguagem funcional que gera
\textit{bytecodes} Java.
A interoperabilidade entre linguagens � de grande interesse, como mostra o sucesso das plataformas .NET
e Java, respectivamente da Microsoft e da Sun Microsystems.
Um dos motivos da baixa ado��o de linguagens funcionais � a falta de uma maneira f�cil destas interoperarem
com outras linguagens.
A m�quina virtual Java executa programas bin�rios na forma de \textit{bytecodes} Java. 
Embora esta m�quina virtual tenha sido desenvolvida para a linguagem de programa��o Java, outras linguagens 
de programa��o podem ser executadas sobre esta.
Alguns dos motivos do grande interesse por m�quinas virtuais s�o a explos�o da web, grandes sistemas
desenvolvidos com base em v�rias linguagens e o suporte a c�digo legado. A m�quina virtual resolve estes
problemas, pois � uma plataforma onde v�rias linguagens podem interoperar.
A linguagem funcional definida neste trabalho tem sintaxe similar a de Haskell,
sua ordem de avalia��o � estrita,
� estaticamente tipada, possui infer�ncia de tipos e n�o possui suporte � sobrecarga (\textit{overload}).
\end{abstract}
